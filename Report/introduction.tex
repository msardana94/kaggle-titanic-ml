\section{Introduction}
\vspace{-0.1in}
The sinking of Titanic is the most infamous shipwrecks and shocked the global world and led to better safety regulations for ships. The major reason was not enough lifeboats for passengers and crew. There was both luck in some people surviving but there were some statistics led results which showed some group survived more than other group. The project involved analyzing which class/category of people survived more than others. The heart of the problem lies in the question, which machine learning techniques is to be used for given training data to perform the predictive task which can help us find out the group of people that had more chances of survival over others.

We follow the concept of Exploratory Data Analysis (EDA) to start with getting insights into the training data. We try to extract as much information from various graphics as we can and infer useful information out of it. Python and R, both being computationally strong language are used during the project work. We first try to deduce as much information as we can from the various plots as explained in the next section. Further we used different models to perform the prediction task and come with a good accuracy. We discuss the procedure along with the results in a detailed manner in the following paragraphs.