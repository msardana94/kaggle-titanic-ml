\section{Conclusion}
\vspace{-0.1in}

We have shown that Exploratory Data Analysis (EDA) techniques are very powerful for getting insights into the information contained in large data sets. We visualized different features and formed trees in RF classifier based on the obtained graphics. We were able to get an accuracy of 82\% after extensive search for important features.

Here are some of things we learnt from this project:
\begin{itemize}

\item We got introduced to concepts related to machine learning such as cross validation, precision-recall, out-of-bag error and grid search.
\item Different ways of looking at data and gathering important assumptions for better analysis.
\item Importance of exploratory data analysis in feature and model selection.
\item Some models are parameter intensive (SVM) and some are feature intensive (Random forest).
\item We realized that even simpler model can give better accuracy compared to complex models (SVM)
\end{itemize}

