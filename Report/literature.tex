\section{Literature Review}
\vspace{-0.1in}
This section discusses the work that has been done in this field and the way it has been done. The area of analytics and its application in the field of movies has not been explored much especially the one based on sentiment analysis of public opinion due to the non-availability of large chunks of data until few years back. However, some key findings as discussed below have made a significant impact on the film industry.

The initial piece of work was done to find a correlation between the box-office data and the tweets from Twitter using simple linear regression. \citet{asur2010predicting} discusses some of the regression techniques implemented for predicting the box-office revenue. Time stamped data of tweets was collected for around 84 movies over a span of 3-4 months which included around 10 million tweets. The author initially calculated the rate at which tweets were posted in different time intervals when the topic relating to movie was trending. Linear regression model was applied to fit in different pairs of data to predict the box-office revenue from the tweet rates in future. This served as the initial information for the prediction purpose. Next, the sentiment analysis of the tweets was done before and after the release of movie to see the change in the intensity of emotions which had clear affects on the revenue as well. The stronger the sentiments(positive/negative) on the day of the release and during the opening weekend, the sharper was the change (increase/decrease) in revenue. Furthermore, some movies which had lukewarm response in the opening week but had stronger display of emotions on social media displayed a sharp rise in their revenue in the weeks ahead of the release. 

\citet{sharda2006predicting} attempted to apply neural network based algorithm for prediction of box office revenue. The author used various features for this purpose which he further classified into independent and dependent variables. The features didn't include any public opinion/sentiments as an input variable rather all the features were related to the movie itself. The different features taken into account were Star Cast, Motion Picture Ratings, Genre, Competition, Sequel, Number of screens all of which were independent of each other and the Box-office Revenue as the only dependent variable. This was the first instance of the application of neural networks in the Film industry to predict the outcomes. The author used Multi-Layer Perceptron with 2 hidden layers for this purpose in which he assigned certain discrete values to each feature and corresponding numerical weight was also given. Further, he tested the performance of the neural network using the percent success rate which is the ratio of total correct classifications to total number of samples, averaged for all classes in the classification problem. This approach proved quite successful by giving the revenue of a movie as the function of various variables which helped a movie producer and director in selecting the crew members and planning out the release in way that can be profitable.

In another attempt at predicting the box-office revenue, \citet{moviemod} used a completely different approach at solving some unanswered questions relating to film industry. Markov chain was used for estimating the revenue based on the consumer behavior. The paper was published more than a decade back when there was no online social media platform for the prediction purpose. The states of the Markov change represented the behavioral states of a consumer about movie which included 'undecided', 'considerer', 'rejecter' etc. The change of state usually take place based on the movie related factors. The probability of moving from one state to other was calculated based on various factors that tend a consumer to form an impression about it. They took into account the most important factor that was key to the success/failure of a movie, vis-a-vis word-of-mouth which nowadays reflect on the social media platforms in the form of user's opinion. The results were significant if we see it relatively as there was no strong mode of communication at that point of time. 